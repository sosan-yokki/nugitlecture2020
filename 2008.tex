
\documentclass[dvipdfmx]{jsarticle}
\begin{document}

2008年午後の部大問3

\vspace{\baselineskip}

 
 解答
 
 (1)
 
$f(z)=\frac{1-e^{2iz}}{z^{2}}$とする。$e^{2iz}=1+\sum_{k=1}^{\infty}$$\frac{(2iz)^{k}}{k!}$である事を用いれば、

$f(z)=-\frac{2i}{z}+\sum_{k=0}^{\infty}$$\frac{i^{k}2^{k+2}}{(k+2)!}$$z^{k}$となる。

(2)

$C_{R}$={$Re^{i\theta}$|$0\leq{\theta}\leq{\pi}$}とする。

|$\int_{C_{R}}$$\frac{1-e^{2iz}}{z^{2}}$$dz$|$\leq$$\int_{0}^{\pi}$$\frac{|1-e^{2iRe^{i\theta}}|}{|R^{2}e^{2i\theta}|}$$|iRe^{i\theta}|$$d\theta$

$\leq$$\frac{1}{R}$$\int_{0}^{\pi}$$(1+|e^{2iRe^{i\theta}}|)$$d\theta$=$\frac{\pi}{R}+$$\frac{1}{R}$$\int_{0}^{\pi}$$|e^{-2Rsin\theta}|d\theta$$\leq$$\frac{2\pi}{R}$$\rightarrow{0}$

ゆえに、$\int_{C_{R}}$$\frac{1-e^{2iz}}{z^{2}}$$dz$$\rightarrow{0}$

(3)

(1)よりRes(0)=-2iより求める値は$\pi$$iRes(0)$=$2\pi$

(4)

$L_{1}=${t|$-R\leq{t}\leq{-r}$}、$L_{1}=${t|$r\leq{t}\leq{R}$}とする。

$C=L_{1}+L_{2}+\Gamma_{r}+C_{R}$とする。

$\int_{C}$$\frac{1-e^{2iz}}{z^{2}}$$dz$=$2\pi$$iRes(0)$=$4\pi$

$\int_{C}$$\frac{1-e^{2iz}}{z^{2}}$$dz$=$\int_{L_{1}+L_{2}+\Gamma_{r}+C_{R}}$$\frac{1-e^{2iz}}{z^{2}}$$dz$

(2)(3)より$2\pi=\int_{-\infty}^{\infty}$$\frac{1-e^{2ix}}{x^{2}}$$dx$  

実部と虚部を比較して$\int_{-\infty}^{\infty}$$\frac{1-cos2x}{x^{2}}$$dx$=$2\pi$だから、$\int_{0}^{\infty}$$\frac{1-cos2x}{x^{2}}$$dx$=$\pi$

$\int_{0}^{\infty}$$\frac{sin^{2}x}{x^{2}}$=$\frac{1}{2}$$\int_{0}^{\infty}$$\frac{1-cos2x}{x^{2}}$$dx$=$\frac{\pi}{2}$








\end{document}
