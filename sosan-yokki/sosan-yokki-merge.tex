\documentclass[../main]{subfiles}
\begin{document}
\title{名古屋大学2016年度理系数学第4問}
\date{}
\maketitle
\section{問題文}
次の問に答えよ.ただし2次方程式の重解は2つと数える.
\subsection*{問1}
次の3つの条件をすべて満たす整数$a$,$b$,$c$,$d$,$e$,$f$を全て求めよ.
\begin{itemize}
\item 2次方程式$x^{2}+ax+b=0$の解は$c$,$d$である
\item 2次方程式$x^{2}+cx+d=0$の解は$e$,$f$である
\item 2次方程式$x^{2}+ex+f=0$の解は$a$,$b$である
\end{itemize}
\subsection*{問2}
2つの数列$\{a_n\}$,$\{b_n\}$は,全ての正の整数$n$について,整数かつ2次方程式$x^{2}+a_{n}x+b_{n}=0$の解が$a_{n+1}$,$b_{n+1}$になっているとする.このとき,
\subsubsection*{(ア)}$|b_m|=|b_{m+1}|=|b_{m+2}|=\dots$となる正の整数$m$が存在することを示せ
\subsubsection*{(イ)}考えられる数列$\{a_n\}$,$\{b_n\}$の組をすべて求めよ.
\section{解答}
\subsection*{問1}
解と係数の関係より以下の方程式が導かれる.
\begin{align}
a&=-c-d\\
b&=cd\\
c&=-e-f\\
d&=ef\\
e&=-a-b\\
f&=ab
\end{align}
(2),(4),(6)から$bdf=cdefab$すなわち,$bdf(1-ace)=0$であることがわかる.場合分けして考える.まず$bdf=0$の場合を考える.このときは$b=0$または$d=0$または$f=0$である.$b=0$とする.(5)に代入して$e=-a$,(6)に代入して$f=0$を得る.$e$と$f$を(3)に代入すると$c=a$,(4)に代入すると$d=0$であることがわかる.さらに$c$と$d$を(1)に代入すると$a=-a$であるから,$a=0$である.以上のことから$(a,b,c,d,e,f)=(0,0,0,0,0,0)$だ.$d=0$,$f=0$である場合も方程式の対称性から同じ結果になる.\\
 次に$1-ace=0$の場合を考える.このとき$ace=1$である,$a$,$c$,$e$はすべて整数ゆえ
\begin{equation*}
(a,c,e)=(1,-1,-1),(-1,1,-1),(-1,-1,1),(1,1,1)
\end{equation*}
の4通りが考えられる.それぞれについて調べる.$(a,c,e)=(1,-1,-1)$のとき(5)から$b=0$ゆえ$(a,b,c,d,e,f)=(0,0,0,0,0,0)$である.$(a,c,e)=(-1,1,-1)$のとき(1)から$d=0$ゆえ$(a,b,c,d,e,f)=(0,0,0,0,0,0)$である.$(a,c,e)=(-1,-1,1)$のとき(3)から$f=0$ゆえ$(a,b,c,d,e,f)=(0,0,0,0,0,0)$である.$(a,c,e)=(1,1,1)$のときはこれらを(1),(3),(5)に代入すると$b=d=f=-2$であるから,$(a,b,c,d,e,f)=(1,-2,1,-2,1,-2)$である.\\
 以上のことから,$(a,b,c,d,e,f)=(0,0,0,0,0,0),(1,-2,1,-2,1,-2)$.
\subsection*{問2}
\subsubsection*{(ア)}
まず数列$\{b_n\}$が全ての正の整数$n$について,0のときは$|b_1|=|b_2|=|b_3|=\dots=0$ゆえ,$m=1$とすればよい.\\
 次に$b_\alpha\ne0$となる正の整数$\alpha$が存在する場合を考える.$x^{2}+a_{\alpha}x+b_{\alpha}=0$の解が$a_{\alpha+1}$,$b_{\alpha+1}$になっているので解と係数の関係から
\begin{align}
a_{\alpha}&=-a_{\alpha+1}-b_{\alpha+1}\\
b_{\alpha}&=a_{\alpha+1}b_{\alpha+1}
\end{align}
である.(8)について両辺絶対値をとると$|b_{\alpha}|=|a_{\alpha+1}||b_{\alpha+1}|$であるが,$|b_{\alpha}|\ne0$ゆえ$|a_{\alpha+1}|\ne0$かつ$|b_{\alpha+1}|\ne0$.よって両辺を$|b_{\alpha+1}|$で割ってかまわない.さらに,$a_{\alpha+1}$が整数なので$|a_{\alpha+1}|\ge1$である.以上のことをまとめると,$\dfrac{|b_{\alpha}|}{|b_{\alpha+1}|}\ge1$すなわち$|b_{\alpha}
|\ge|b_{\alpha+1|}$である.$|b_{\alpha+1}|\ne0$なので$b_{\alpha+1}\ne0$であるから,同様の議論を経て$|b_{\alpha+1}
|\ge|b_{\alpha+2|}$と,$b_{\alpha+2}\ne0$を得る.同様に繰り返すと
\begin{align}
|b_\alpha|\ge|b_{\alpha+1}|\ge|b_{\alpha+2}|\dots\\
\intertext{である.さらに$n\le\alpha$で$b_n$は整数かつ$|b_n|\ne0$なので,この場合$|b_n|$は$1$,$2$,\dots$|b_\alpha|$の$|b_\alpha|$通りの値しかとれない.ゆえに(9)と部屋割り論法からある整数$m$が存在して}
0\ne|b_m|=|b_{m+1}|=|b_{m+2}|\dots
\end{align}
\subsubsection*{(イ)}
(ア)と同じ場合分けをする.まず数列$\{b_n\}$が全ての正の整数$n$について,0とする.全ての正の整数$n$について,$x^{2}+a_{n}x+b_{n}=0$の解が$a_{n+1}$,$b_{n+1}$なので解と係数の関係から$a_n=-a_{n+1}-b_{n+1}=-a_{n+1}$である.よって$\{a_n\}$は初項$a_1$公比$-1$の等比数列である.$a_1$は整数値である以外の条件は無い.ゆえに$a_n=a(-1)^{n-1}$(ただし$a$は任意の整数)である.\\
 次に$b_\alpha\ne0$となる正の整数$\alpha$が存在する場合を考える.(ア)の議論よりある整数$m$が存在して(10)が成り立つ.$x^{2}+a_mx+b_m=0$の解が$a_{m+1}$,$b_{m+1}$になっているので解と係数の関係から
\begin{align}
a_{m}&=-a_{m+1}-b_{m+1}\\
b_{m}&=a_{m+1}b_{m+1}
\end{align}
(12)より$|b_m|=|a_{m+1}||b_{m+1}|=|a_{m+1}||b_m|$なので両辺$|b_m|\ne0$で割って$|a_{m+1}|=1$を得る.添え字を1つずつ増やすことで$a_{m+2}$,$a_{m+3},\dots$についても同様なことがいえるので,
\begin{align}
1=|a_{m+1}|=|a_{m+2}|=|a_{m+3}|\dots
\end{align}
である.(11)と(12)は添え字を1つ増やしても成り立つので,
\begin{align}
a_{m+1}&=-a_{m+2}-b_{m+2}\\
b_{m+1}&=a_{m+2}b_{m+2}
\end{align}
(10)と(14)から$0\ne b_{m+2}=-(a_{m+1}+a_{m+2})$であり(13)から$a_{m+1},a_{m+2}=\pm1$なので,$(a_{m+1},a_{m+2})=(1,1),(-1,-1)$の2つが考えられる.それぞれ条件に合うか調べる.\\
 $(a_{m+1},a_{m+2})=(1,1)$の場合(14)より$b_{m+2}=-2$,(15)より$b_{m+1}=-2$.2次方程式$x^{2}+a_{m+2}x+b_{m+2}=x^{2}+x-2=0$の解が$a_{m+3}$,$b_{m+3}$になっていればよい.$x^{2}+x-2=0$の解は$x=1,-2$である.条件(10)から$a_{m+3}=1$,条件(13)から$b_{m+3}=-2$となる.
 $(a_{m+1},a_{m+2})=(-1,-1)$の場合(14)より$b_{m+2}=2$,(15)より$b_{m+1}=-2$.2次方程式$x^{2}+a_{m+2}x+b_{m+2}=x^{2}-x+2=0$の解が$a_{m+3}$,$b_{m+3}$になっていればよい.しかし$x^{2}-x+2=0$の解は$x=-1\pm\sqrt{7}i$であるから整数であるという条件に合わない.\\
 以上のことから$a_{m+1}=a_{m+2}=a_{m+3}=1$,$b_{m+1}=b_{m+2}=b_{m+3}=-2$である.(14)と(15)の添え字を1つずつ増やすと同様にして,$a_{m+2}=a_{m+3}=a_{m+4}=1$,$b_{m+2}=b_{m+3}=b_{m+4}=-2$である.これを繰り返すと$n\ge m+1$で$a_n=1$,$b_n=-2$である.一方(11),(12)に$a_{m+1}=1$と$b_{m+1}=-2$を代入すると$a_m=1$と$b_m=-2$を得る.(11)と(12)の添え字を1つ減らすことで同様にして,$a_{m-1}=1$,$b_{m-1}=-2$である.これを繰り返すと$n\le m$で$a_n=1$,$b_n=-2$である.ゆえに任意の正の整数$n$について$a_n=1$,$b_n=-2$である.\\
 よって2つの数列$\{a_n\}$,$\{b_n\}$の組は,以下の2通りのみである.
\begin{itemize}
\item $a_n=a(-1)^{n-1},b_n=0$(ただし$a$は任意の整数)
\item $a_n=1,b_n=-2$
\end{itemize}
\begin{thebibliography}{9}
\bibitem{kyo-gaku}大竹真一,名古屋大の理系数学15ヶ年[第5版],教学社,2009,pp52-54
\end{thebibliography}
\end{document}